%\documentclass[11pt]{book}
%\usepackage{palatino}
%\usepackage{amsfonts,amsmath,amssymb}
%
%\begin{document}
%\pagestyle{empty}
%{\noindent\bf Spring 2022 \hfill Firstname M.~Lastname}
%\vskip 16pt
%\centerline{\bf University of Central Florida}
%\centerline{\bf College of Business}
%\vskip 16pt
%\centerline{\bf QMB 6912}
%\centerline{\bf Capstone Project in Business Analytics}
%\vskip 10pt
%\centerline{\bf Solutions:  Problem Sets \#1 \& 2}
%\vskip 32pt
%
%This example has sections for each article in Problem Set \#1.

\section{Introduction}

In this paper I analyze the prices from sales of used tractors. 
My aim is to investigate two questions.
First, I intend to investigate the premium attached to the John Deere brand of tractors. 
If there is such a premium, how high is it? 
Also, is such a premium related to particular models, 
or is there a relationship with the rate of depreciation of these tractors 
compared to that of other brands?
Second, if I were selling a used tractor, should I wait until a particular season to sell it?
Does the price change depending on the season? 

In the following pages, I will investigate these questions and
ultimately fit a model for the price of used tractors, 
allowing for endogeneity in the choice of the
characteristics of tractors produced and sold by each manufacturer. 
To understand the dynamics of a sample selection model, 
it is worthwhile to understand the research related to these questions. 



\section{Economic Theory}

\subsection{Market for Lemons}

Akerlof's description of the market for lemons is a long-standing contribution to the understanding of asymmetric information. 


\subsection{Characteristic Theory}

The paper by Lancaster (1966) is a contribution to economic theory in which he makes the case for a sound theory of consumer choice, 
in which the consumer's preferences are defined on the characteristics of different goods, 
rather than the quantities of uniformly-defined goods.


\subsection{Hedonic Pricing Models}

Rosen (1974) builds on the work of Lancaster (1966)
by providing an empirical framework for estimating the value
of products based on their characteristics. 



\section{Empirical Framework}

\subsection{Tobit Models}

Heckman (1979) described sample selection models as a model specification question. 

Amemiya (1984) summarized the variaous types of Tobit models
and provides a review of research with applications of these models. 

Lee and Trost (1978) is a well-known application to housing markets, 
which incorporates the fact that the residents make a decision
to either own or rent a home before buying. 


% \end{document}