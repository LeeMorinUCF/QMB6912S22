\documentclass[11pt]{paper}
\usepackage{fullpage}
\usepackage{palatino}
\usepackage{amsfonts,amsmath,amssymb}
% \usepackage{graphicx}

\usepackage{listings}
\usepackage{textcomp}
\usepackage{color}

\definecolor{dkgreen}{rgb}{0,0.6,0}
\definecolor{gray}{rgb}{0.5,0.5,0.5}
\definecolor{mauve}{rgb}{0.58,0,0.82}

\lstset{frame=tb,
  language=R,
  aboveskip=3mm,
  belowskip=3mm,
  showstringspaces=false,
  columns=flexible,
  basicstyle={\small\ttfamily},
  numbers=none,
  numberstyle=\tiny\color{gray},
  keywordstyle=\color{blue},
  commentstyle=\color{dkgreen},
  stringstyle=\color{mauve},
  breaklines=true,
  breakatwhitespace=true,
  tabsize=3
}



\ifx\pdftexversion\undefined
    \usepackage[dvips]{graphicx}
\else
    \usepackage[pdftex]{graphicx}
    \usepackage{epstopdf}
    \epstopdfsetup{suffix=}
\fi

\usepackage{subfig}


% This allows pdflatex to print the curly quotes in the
% significance codes in the output of the GAM.
\UseRawInputEncoding

\begin{document}

%%%%%%%%%%%%%%%%%%%%%%%%%%%%%%%%%%%%%%%%
% Problem Set 7
%%%%%%%%%%%%%%%%%%%%%%%%%%%%%%%%%%%%%%%%

\pagestyle{empty}
{\noindent\bf Spring 2021 \hfill Firstname M.~Lastname}
\vskip 16pt
\centerline{\bf University of Central Florida}
\centerline{\bf College of Business}
\vskip 16pt
\centerline{\bf QMB 6911}
\centerline{\bf Capstone Project in Business Analytics}
\vskip 10pt
\centerline{\bf Solutions:  Problem Set \#10}
\vskip 32pt
\noindent
% 
\section{Data Description}

This analysis follows the script \texttt{Tractor\_Reg\_Model.R} to produce a more accurate model for used tractor prices with the data from \texttt{TRACTOR7.csv} in the \texttt{Data} folder. 
The dataset includes the following variables.
\begin{table}[h!]
\begin{tabular}{l l l}

$saleprice_i$ & = & the price paid for tractor $i$ in dollars \\
% 
$horsepower_i$ & = & the horsepower of tractor $i$ \\
$age_i$ & = & the number of years since tractor $i$ was manufactured  \\
$enghours_i$ & = & the number of hours of use recorded for tractor $i$  \\
$diesel_i$ & = & an indicator of whether tractor $i$ runs on diesel fuel \\ %, $0$ otherwise \\
$fwd_i$ & = & an indicator of whether tractor $i$ has four-wheel drive \\ %, $0$ otherwise \\
$manual_i$ & = & an indicator of whether tractor $i$ has a manual transmission \\ %, $0$ otherwise \\
$johndeere_i$ & = & an indicator of whether tractor $i$ is manufactured by John Deere \\ %, $0$ otherwise \\
$cab_i$ & = & an indicator of whether tractor $i$ has an enclosed cab \\ %, $0$ otherwise \\
% 
$spring_i$ & = & an indicator of whether tractor $i$ was sold in April or May \\ %, $0$ otherwise \\
$summer_i$ & = & an indicator of whether tractor $i$ was sold between June and September \\ %, $0$ otherwise \\
$winter_i$ & = & an indicator of whether tractor $i$ was sold between December and March \\ %, $0$ otherwise \\

\end{tabular}
\end{table}
%

I will revisit the recommended linear model
from Problem Set \#7, 
which was supported in
Problem Sets \#8 and  \#9 
by considering other nonlinear specifications
within a Generalized Additive Model. 




Then I will further investigate this nonlinear relationship
by considering the issue of sample selection:
John Deere may produce tractors of specific qualities based on
their perceived value to typical John Deere customers, 
in ways that are not represented by the variables in the dataset.



%%%%%%%%%%%%%%%%%%%%%%%%%%%%%%%%%%%%%%%%
\clearpage
\section{Linear Regression Model}
%%%%%%%%%%%%%%%%%%%%%%%%%%%%%%%%%%%%%%%%

A natural staring point is the recommended linear model
from Problem Set \#7. 

\subsection{Quadratic Specification for Horsepower}

In the demo for Problem Set \#7, 
we considered the advice of
a used tractor dealer who reported that overpowered used tractors are hard to sell, since they consume more fuel. 
This implies that tractor prices often increase with horsepower, up to a point, but beyond that they decrease. 
To incorporate this advice, I created and included a variable for squared horsepower. 
A decreasing relationship for high values of horsepower
is characterized by 
a positive coefficient on the horsepower variable and
a negative coefficient on the squared horsepower variable. 

% 

\begin{table}
\begin{center}
\begin{tabular}{l c c}
\hline
 & Model 1 & Model 2 \\
\hline
(Intercept)         & $8.72792^{***}$  & $8.72792^{***}$  \\
                    & $(0.10602)$      & $(0.07427)$      \\
horsepower          & $0.01112^{***}$  & $0.01112^{***}$  \\
                    & $(0.00107)$      & $(0.00075)$      \\
squared\_horsepower & $-0.00001^{***}$ & $-0.00001^{***}$ \\
                    & $(0.00000)$      & $(0.00000)$      \\
age                 & $-0.03233^{***}$ & $-0.03233^{***}$ \\
                    & $(0.00358)$      & $(0.00251)$      \\
enghours            & $-0.00004^{***}$ & $-0.00004^{***}$ \\
                    & $(0.00001)$      & $(0.00001)$      \\
diesel              & $0.20350^{*}$    & $0.20350^{**}$   \\
                    & $(0.09805)$      & $(0.06869)$      \\
fwd                 & $0.26539^{***}$  & $0.26539^{***}$  \\
                    & $(0.05820)$      & $(0.04077)$      \\
manual              & $-0.15015^{*}$   & $-0.15015^{***}$ \\
                    & $(0.06189)$      & $(0.04336)$      \\
johndeere           & $0.31872^{***}$  & $0.31872^{***}$  \\
                    & $(0.07186)$      & $(0.05034)$      \\
cab                 & $0.48345^{***}$  & $0.48345^{***}$  \\
                    & $(0.07003)$      & $(0.04906)$      \\
\hline
R$^2$               & $0.80591$        & $0.80591$        \\
Adj. R$^2$          & $0.79935$        & $0.80269$        \\
Num. obs.           & $276$            & $552$            \\
\hline
\multicolumn{3}{l}{\scriptsize{$^{***}p<0.001$; $^{**}p<0.01$; $^{*}p<0.05$}}
\end{tabular}
\caption{Quadratic Model for Tractor Prices}
\label{tab:reg_sq_horse}
\end{center}
\end{table}

% 
The results of this regression specification are shown in 
Table \ref{tab:reg_sq_horse}. 
%
The squared horsepower variable has a coefficient of $-2.081e-05$, which is nearly ten times as large as the standard error of $2.199e-06$, which is very strong evidence against the null hypothesis of a positive or zero coefficient. 
I conclude that the log of the sale price does decline for large values of horsepower. 


With the squared horsepower variable, the $\bar{R}^2$ is $0.764$, indicating that it is a much stronger model than the others we considered. 
The $F$-statistic is large, indicating that it is a better candidate than the simple average log sale price. 
The new squared horsepower variable is statistically significant and the theory behind it is sound, since above a certain point, added horsepower may not improve performance but will cost more to operate. 
This new model is much improved over the previous models with a linear specification for horsepower.
Next, I will attempt to improve on this specification, 
as we did for Problem Set \#8. 





%%%%%%%%%%%%%%%%%%%%%%%%%%%%%%%%%%%%%%%%
\clearpage
\section{Sample Selection}
%%%%%%%%%%%%%%%%%%%%%%%%%%%%%%%%%%%%%%%%


% \clearpage
\subsection{Predicting the Selection into Samples}


The specification in 
Table \ref{tab:reg_sq_horse}
assumes a quadratic functional form for
the relationship between price and horsepower, 
without selecting into samples by brand.
% 
To investigate this relationship further, 
consider the set of variables that are related to
whether or not John Deere makes a tractor
with the characteristics observed in the dataset. 


% \clearpage
\subsection{Estimating a Sample Selection Model}



%%%%%%%%%%%%%%%%%%%%%%%%%%%%%%%%%%%%%%%%
\end{document}
%%%%%%%%%%%%%%%%%%%%%%%%%%%%%%%%%%%%%%%%
