\documentclass[11pt]{paper}
\usepackage{fullpage}
\usepackage{palatino}
\usepackage{amsfonts,amsmath,amssymb}
% \usepackage{graphicx}

\usepackage{listings}
\usepackage{textcomp}
\usepackage{color}

\definecolor{dkgreen}{rgb}{0,0.6,0}
\definecolor{gray}{rgb}{0.5,0.5,0.5}
\definecolor{mauve}{rgb}{0.58,0,0.82}

\lstset{frame=tb,
  language=R,
  aboveskip=3mm,
  belowskip=3mm,
  showstringspaces=false,
  columns=flexible,
  basicstyle={\small\ttfamily},
  numbers=none,
  numberstyle=\tiny\color{gray},
  keywordstyle=\color{blue},
  commentstyle=\color{dkgreen},
  stringstyle=\color{mauve},
  breaklines=true,
  breakatwhitespace=true,
  tabsize=3
}



\ifx\pdftexversion\undefined
    \usepackage[dvips]{graphicx}
\else
    \usepackage[pdftex]{graphicx}
    \usepackage{epstopdf}
    \epstopdfsetup{suffix=}
\fi

\usepackage{subfig}



% Trying different tips from Google help to get a summary to print.
% \usepackage[T1]{fontenc} 
% \usepackage[utf8]{inputenc}
% 
% \usepackage[utf8]{inputenc}
% \usepackage[utf8x]{inputenc}
\UseRawInputEncoding
% This allows pdflatex to print the curly quotes in the
% significance codes in the output of the GAM.


\begin{document}

%%%%%%%%%%%%%%%%%%%%%%%%%%%%%%%%%%%%%%%%
% Problem Set 7
%%%%%%%%%%%%%%%%%%%%%%%%%%%%%%%%%%%%%%%%

\pagestyle{empty}
{\noindent\bf Spring 2021 \hfill Firstname M.~Lastname}
\vskip 16pt
\centerline{\bf University of Central Florida}
\centerline{\bf College of Business}
\vskip 16pt
\centerline{\bf QMB 6911}
\centerline{\bf Capstone Project in Business Analytics}
\vskip 10pt
\centerline{\bf Solutions:  Problem Set \#10}
\vskip 32pt
\noindent
% 
% 
\section{Data Description}
% 
By engaging an industry consultant to gather relevant and appropriate 
information, your firm has been able to put together data concerning 248 
different fly-fishing reels, over one-half of which are produced in the 
United States, with the remainder being produced in Asia---either in China 
or Korea.  These data are contained in the file {\tt FlyReels.csv}, which is
available in the {\tt Data} folder.
Each fly-fishing reel in the data set is a row, while the columns correspond 
to the variables whose names and definitions are the following:
\bigskip
\begin{table}[ht]
\centering
\begin{tabular}{ll}
  \hline
    Variable & Definition \\
  \hline

    {\tt Name}        &product name (a string) \\ 
    {\tt Brand}       &brand name (a string) \\ 
    {\tt Weight}      &weight of reel in ounces (a real number) \\ 
    {\tt Diameter}    &diameter of reel in inches (a real number) \\ 
    {\tt Width}       &width of reel in inches (a real number) \\ 
    {\tt Price}       &price of reel in dollars (a real number) \\ 
    {\tt Sealed}      &whether the reel is sealed; {\tt "Yes"} versus
                        {\tt "No"} (a string) \\ 
    {\tt Country}     &country of manufacture, (a string) \\ 
    {\tt Machined}    &whether the reel is machined versus cast;
                        machined={\tt "Yes"}, \\ 
                      &while cast={\tt "No"} (a string) \\ 
  \hline
\end{tabular}
%\caption{Summary of Numeric Variables}
%\label{tab:summary}
\end{table}


I will revisit the recommended linear model
from Problem Set \#7, 
which was supported in
Problem Sets \#8 and  \#9 
by considering other nonlinear specifications
within a Generalized Additive Model. 



Then I will further investigate this nonlinear relationship
by considering the issue of sample selection:
fly reel manufacturers 
may produce 
fly reels in each country
with specific qualities based on
their perceived value to typical 
American 
customers, 
in ways that are not represented by the variables in the dataset.



%%%%%%%%%%%%%%%%%%%%%%%%%%%%%%%%%%%%%%%%
\clearpage
\section{Linear Regression Model}
%%%%%%%%%%%%%%%%%%%%%%%%%%%%%%%%%%%%%%%%

A natural staring point is the recommended linear model
from Problem Set \#7. 

\subsection{Linear Model with \texttt{Sealed*Made\_in\_USA} Interaction}

Last week I investigated whether 
the functional form should include different specifications by
country of manufacture.
% 
The model included the continuous variables 
width, diameter, and density, 
as well as categorical variables for 
country of manufacture, 
and whether or not the reels were sealed or machined. 
% 
In addition to the indicator for the country of manufacture, the model included an indicator for an interaction between
the the country of manufacture indicator and the indicator for whether the reels were sealed or unsealed. 
% 
The dependent variable was chosen as 
the logarithm of the fly reel price, 
since the results were similar to those from the model 
with the optimal Box-Cox transformation, 
without the added complexity. 
% 
The results of this regression specification are shown in 
Table \ref{tab:reg_sealed_USA}. 
% 

\begin{table}
\begin{center}
\begin{tabular}{l c}
\hline
 & Model 1 \\
\hline
(Intercept)                 & $2.00999^{***}$ \\
                            & $(0.26125)$     \\
Width                       & $0.33575^{*}$   \\
                            & $(0.15622)$     \\
Diameter                    & $0.39567^{***}$ \\
                            & $(0.05076)$     \\
Density                     & $1.21296^{***}$ \\
                            & $(0.21948)$     \\
SealedYes                   & $0.62731^{***}$ \\
                            & $(0.08622)$     \\
MachinedYes                 & $0.64934^{***}$ \\
                            & $(0.08320)$     \\
made\_in\_USATRUE           & $0.74633^{***}$ \\
                            & $(0.09247)$     \\
SealedYes:made\_in\_USATRUE & $-0.29519^{**}$ \\
                            & $(0.10092)$     \\
\hline
R$^2$                       & $0.74893$       \\
Adj. R$^2$                  & $0.74160$       \\
Num. obs.                   & $248$           \\
\hline
\multicolumn{2}{l}{\scriptsize{$^{***}p<0.001$; $^{**}p<0.01$; $^{*}p<0.05$}}
\end{tabular}
\caption{Linear Model for Fly Reel Prices}
\label{tab:reg_sealed_USA}
\end{center}
\end{table}

% 
Next, I will attempt to improve on this specification,
using Tobit models for sample selection.





%%%%%%%%%%%%%%%%%%%%%%%%%%%%%%%%%%%%%%%%
\pagebreak
\subsection{Comparison by Country of Manufacture}
%%%%%%%%%%%%%%%%%%%%%%%%%%%%%%%%%%%%%%%%

Table \ref{tab:reg_by_country} shows the results of 
a series of regression models 
on different samples by country of manufacture.
Model 1 shows the results 
for the full model
using the sample of fly reels made in the USA
and Model 3 shows the remaining fly reels made in China or Korea.
Models 2 and 4 show the estimates from a reduced model
on each sample, eliminating the variable \texttt{Width}, 
which was not significant in either of these smaller samples.


\begin{table}
\begin{center}
\begin{tabular}{l c c}
\hline
 & Model 1 & Model 2 \\
\hline
(Intercept) & $3.35^{***}$ & $2.03^{***}$ \\
            & $(0.30)$     & $(0.47)$     \\
Width       & $0.28$       & $0.39$       \\
            & $(0.22)$     & $(0.23)$     \\
Diameter    & $0.44^{***}$ & $0.36^{***}$ \\
            & $(0.07)$     & $(0.08)$     \\
Density     & $1.13^{***}$ & $1.32^{**}$  \\
            & $(0.25)$     & $(0.41)$     \\
SealedYes   & $0.32^{***}$ & $0.65^{***}$ \\
            & $(0.06)$     & $(0.10)$     \\
MachinedYes &              & $0.65^{***}$ \\
            &              & $(0.09)$     \\
\hline
R$^2$       & $0.54$       & $0.75$       \\
Adj. R$^2$  & $0.52$       & $0.73$       \\
Num. obs.   & $135$        & $113$        \\
\hline
\multicolumn{3}{l}{\scriptsize{$^{***}p<0.001$; $^{**}p<0.01$; $^{*}p<0.05$}}
\end{tabular}
\caption{Regression Models by Country of Manufacture}
\label{tab:reg_by_country}
\end{center}
\end{table}


The width of the reel is insignificant in both samples
and the coefficients are qualitatively similar across the samples, as well as matching in significance. 
This suggests that one model might be sufficient. 

%To test this statistically, I conduct an $F$-test. 
%This compares a single model with only an
%indicator for the country of manufacture
%(the restricted model)
%with a separate model for each country.
%In this case, the full, unrestricted model has 
%$K = 2\times6 = 12$ parameters, one for each variable in two models. 
%The test that all of the coefficients are the same has 
%$M = 6 - 1 = 5$
%restrictions. 
%The one restriction fewer accounts for the made-in-USA indicator
%in the full model, 
%which allows for two separate intercepts. 
%% 
%The $F$-statistic has a value of 
%
%$$ 
%\frac{(RSS_M - RSS)/M}{RSS/(N - K - 1)} = \frac{(26.24962 - 25.2235)/5}{25.2235/235} = 1.912007. 
%$$
%
%This value is greater than 1, so we can compare it to the critical value
%of the $F$-statistic at the specified degrees of freedom for
%a conventional level of significance.
%These critical values are 
%3.095, 2.252, and 1.872
%at the 1\%, 5\%, and 10\%
%levels of significance, respectively.
%
%This places the F-statistic between the critical values for the
%5 and 10 percent levels of significance.
%Conclude that fly reel prices may have some difference by
%country of manufacture but the difference is marginal.
%This suggests little justification for separate models by
%country of manufacture.
%We can investigate small differences between the models.
%% 
%Next, we will turn this model around
%to determine whether features of the fly reels
%are useful to predict country of manufacture. 
%This will help determine whether manufacturers tend to 
%jointly choose fly reel designs and country of manufacture. 



%%%%%%%%%%%%%%%%%%%%%%%%%%%%%%%%%%%%%%%%
\clearpage
\section{Sample Selection}
%%%%%%%%%%%%%%%%%%%%%%%%%%%%%%%%%%%%%%%%


% \clearpage
\subsection{Predicting the Selection into Samples}


The specification in 
Table \ref{tab:reg_sealed_USA}
assumes a linear functional form for
the relationship between characteristics and prices of fly reels, 
without selecting into samples by brand.
% 
To investigate this relationship further, 
consider the set of variables that are related to
whether or not 
a manufacturer decides to manufacture fly reels in 
America or overseas
with the characteristics observed in the dataset. 


\begin{table}
\begin{center}
\begin{tabular}{l c c}
\hline
 & Model 1 & Model 2 \\
\hline
(Intercept)    & $-3.46545$       & $3.38980^{**}$   \\
               & $(159.27332)$    & $(1.06142)$      \\
Weight         & $0.11330$        & $0.13348^{*}$    \\
               & $(0.19412)$      & $(0.05711)$      \\
Diameter       & $-0.08526$       &                  \\
               & $(0.86262)$      &                  \\
Width          & $-0.71463$       & $-2.45931^{**}$  \\
               & $(1.68668)$      & $(0.81177)$      \\
Volume         & $-0.03294$       &                  \\
               & $(0.09631)$      &                  \\
Density        & $-2.19723$       & $-2.02613^{*}$   \\
               & $(2.46062)$      & $(0.98455)$      \\
SealedYes      & $-1.23411^{***}$ & $-0.68650^{***}$ \\
               & $(0.23449)$      & $(0.18639)$      \\
MachinedYes    & $6.58401$        &                  \\
               & $(159.21535)$    &                  \\
\hline
AIC            & $267.46344$      & $328.68383$      \\
BIC            & $295.57087$      & $346.25098$      \\
Log Likelihood & $-125.73172$     & $-159.34192$     \\
Deviance       & $251.46344$      & $318.68383$      \\
Num. obs.      & $248$            & $248$            \\
\hline
\multicolumn{3}{l}{\scriptsize{$^{***}p<0.001$; $^{**}p<0.01$; $^{*}p<0.05$}}
\end{tabular}
\caption{Probit Models for Country-of-Manufacture Selection of Fly Reels}
\label{tab:reg_probit}
\end{center}
\end{table}


Table \ref{tab:reg_probit} 
shows the estimates for a probit model to predict the selection
into samples by country of manufacture.
% 
Model 1 in Table \ref{tab:reg_probit} 
shows a preliminary probit model to predict the selection indicator,
with all the other explanatory variables in the model.
American fly reel manufacture seems only to be related to 
whether or not the reels are sealed.
% 
Model 2 shows the result of a variable-reduction exercise
to eliminate variables that are not statistically significant.
These estimates provide a concise but useful model to
indicate the fly reel designs that manufacturers would 
prefer to manufacture on American soil.
% 

% 
This model is used to specify the selection equation
of the sample selection estimates discussed next. 


% \clearpage
\subsection{Estimating a Sample Selection Model}

Table \ref{tab:tobit_5_sel} shows the estimates from a model that accounts for sample selection. 
The models are estimates from the Tobit model of type 5, 
which is a model specification that allows for switching 
of the observations in the sample into two models:
one for the value of fly reels made in the USA 
and the other for fly reels produced elsewhere.
% 

Each column shows the estimates from a separate model
and the series of models is the result of a downward selection
procedure in which a statistically insignificant variable
was removed from each of the previous models in the sequence. 
% 
In each model, the estimates are grouped into three categories.
% 
The first block of coefficients describe the selection model
to determine whether a fly reel design would be manufactured in the USA.
These coefficients are denoted by the prefix``S:''.
Below these lies two blocks of coefficients for the observation equations. 
The notation ``O: <name of variable> (i)''
indicates the coefficient for the particular variable
in the observation equation for sample $i$. 
In this specification, the first observation equation represents
fly reels made overseas (\texttt{johndeere == 0}), 
while equation 2 represents the fly reels made in the USA
(\texttt{johndeere == 1}).


Model 1 shows the estimates from the full model. 
Several of the coefficients in the model are 
statistically insignificant and the model has numerical issues.
In particular, some of the standard errors are undefined,
evidenced by the missing standard error for \texttt{sigma1}.
This suggests that the likelihood function is flat in some areas
of the parameter space.
%
This model has its imperfections but is a good start.
Several variables are statistically insignificant
but these can be removed one by one
to produce a refined model.
%
The goal will be to obtain a final model that has
well-defined standard errors for all variables
and, ideally, all coefficients statistically significant.


Model 2 shows the estimates from a reduced model,
eliminate \texttt{Diameter} from the other-country equation.
There still exists coefficients that are statistically insignificant
and the numerical issue remains for \texttt{sigma1}. 
One step further, Model 3
eliminates \texttt{Width} from the American observation equation.
Again, this model is an improvement but a numerical issue remains. 

Model 4 excludes \texttt{Weight} from the selection equation.
The produces a set of estimates that are numerically stable, 
within a strictly concave region of the likelihood function. 
This model, however, still contains
some variables that are statistically significant.
Model 5 excludes \texttt{Density} from the selection equation.
Again, this model is well-behaved numerically
but one statistically insignificant variable remains.

The next step is to estimate what would be Model 6
by eliminating \texttt{Width} from the selection equation.
This specification is problematic, however, 
as it produces an error message indicating severe multicollinearity. 
% 
With these numerical problems,
it is better to keep the additional variable
in the selection equation,
even though it may be statistically insignificant.
Notice that the remaining selection variable \texttt{Sealed}
appears in both observation equations and there is no
variable in the selection equation that is excluded.
The other extreme would offer better performance,
that is, having some variables in the selection
equation that are not included in the observation equations.



\begin{table}
\begin{center}
\begin{small}
\begin{tabular}{l c c c c c}
\hline
 & Model 1 & Model 2 & Model 3 & Model 4 & Model 5 \\
\hline
S: (Intercept)     & $3.38999^{*}$   & $3.38995^{*}$   & $3.39163^{**}$   & $1.84476^{*}$    & $1.06496^{*}$    \\
                   & $(1.52501)$     & $(1.39731)$     & $(1.24301)$      & $(0.88778)$      & $(0.49241)$      \\
S: Width           & $-2.45920$      & $-2.45922^{*}$  & $-2.45815^{*}$   & $-0.87031$       & $-0.59098$       \\
                   & $(1.26540)$     & $(1.18220)$     & $(1.06044)$      & $(0.51987)$      & $(0.45219)$      \\
S: Weight          & $0.13399$       & $0.13392$       & $0.13475$        &                  &                  \\
                   & $(0.08555)$     & $(0.08136)$     & $(0.07909)$      &                  &                  \\
S: Density         & $-2.02599$      & $-2.02603$      & $-2.02510$       & $-0.88282$       &                  \\
                   & $(1.36507)$     & $(1.18246)$     & $(1.15198)$      & $(0.83583)$      &                  \\
S: SealedYes       & $-0.68657^{**}$ & $-0.68656^{**}$ & $-0.68891^{***}$ & $-0.52031^{**}$  & $-0.47216^{**}$  \\
                   & $(0.25772)$     & $(0.21197)$     & $(0.15923)$      & $(0.17879)$      & $(0.17222)$      \\
O: (Intercept) (1) & $0.26764$       & $0.23849$       & $0.30737$        & $1.89824^{**}$   & $2.14641^{***}$  \\
                   & $(1.08958)$     & $(0.88866)$     & $(0.78180)$      & $(0.59989)$      & $(0.50263)$      \\
O: Width (1)       & $1.73696^{***}$ & $1.75076^{***}$ &                  &                  &                  \\
                   & $(0.51414)$     & $(0.50948)$     &                  &                  &                  \\
O: Diameter (1)    & $0.02868$       &                 &                  &                  &                  \\
                   & $(0.18074)$     &                 &                  &                  &                  \\
O: Density (1)     & $1.70252$       & $1.70122$       & $1.73905^{*}$    & $1.09840^{*}$    & $0.79690^{*}$    \\
                   & $(0.97121)$     & $(0.88239)$     & $(0.79321)$      & $(0.48704)$      & $(0.37891)$      \\
O: SealedYes (1)   & $1.22730^{***}$ & $1.23870^{***}$ & $1.24406^{***}$  & $0.92288^{***}$  & $0.90755^{***}$  \\
                   & $(0.21538)$     & $(0.22743)$     & $(0.18759)$      & $(0.11757)$      & $(0.11456)$      \\
O: MachinedYes     & $0.61018^{**}$  & $0.60781^{**}$  & $0.63222^{***}$  & $0.75933^{***}$  & $0.76203^{***}$  \\
                   & $(0.21795)$     & $(0.20627)$     & $(0.18592)$      & $(0.09168)$      & $(0.09128)$      \\
O: (Intercept) (2) & $3.32465^{***}$ & $3.32465^{***}$ & $3.30003^{***}$  & $3.49131^{***}$  & $3.44177^{***}$  \\
                   & $(0.30694)$     & $(0.30651)$     & $(0.30380)$      & $(0.28482)$      & $(0.27958)$      \\
O: Width (2)       & $0.14939$       & $0.14939$       &                  &                  &                  \\
                   & $(0.26002)$     & $(0.25871)$     &                  &                  &                  \\
O: Diameter (2)    & $0.47103^{***}$ &                 &                  &                  &                  \\
                   & $(0.07550)$     &                 &                  &                  &                  \\
O: Density (2)     & $1.09216^{***}$ & $1.09216^{***}$ & $1.04813^{***}$  & $1.00088^{***}$  & $1.06657^{***}$  \\
                   & $(0.26518)$     & $(0.26347)$     & $(0.26497)$      & $(0.26029)$      & $(0.24389)$      \\
O: SealedYes (2)   & $0.26663^{**}$  & $0.26662^{**}$  & $0.21502^{***}$  & $0.27774^{***}$  & $0.28495^{***}$  \\
                   & $(0.08286)$     & $(0.08173)$     & $(0.06041)$      & $(0.07395)$      & $(0.07554)$      \\
sigma1             & $0.83812$       & $0.89629$       & $0.80395$        & $0.52930^{***}$  & $0.53170^{***}$  \\
                   & $$              & $$              & $$               & $(0.07186)$      & $(0.06979)$      \\
sigma2             & $0.31216^{***}$ & $0.31216^{***}$ & $0.33612^{***}$  & $0.31969^{***}$  & $0.31675^{***}$  \\
                   & $(0.04046)$     & $(0.04029)$     & $(0.02417)$      & $(0.04534)$      & $(0.04844)$      \\
rho1               & $-1.00000$      & $-0.99511$      & $-1.00000$       & $-0.87806^{***}$ & $-0.88135^{***}$ \\
                   & $(3.66988)$     & $$              & $$               & $(0.08675)$      & $(0.08048)$      \\
rho2               & $0.39798$       & $0.39798$       & $0.75807^{***}$  & $0.46352$        & $0.43112$        \\
                   & $(0.44205)$     & $(0.43892)$     & $(0.06565)$      & $(0.40092)$      & $(0.48226)$      \\
O: Diameter        &                 & $0.47102^{***}$ & $0.50888^{***}$  & $0.47502^{***}$  & $0.48018^{***}$  \\
                   &                 & $(0.07531)$     & $(0.05880)$      & $(0.05735)$      & $(0.05664)$      \\
O: Width           &                 &                 & $1.81227^{***}$  & $1.25522^{***}$  & $1.17131^{***}$  \\
                   &                 &                 & $(0.46808)$      & $(0.30287)$      & $(0.28044)$      \\
\hline
AIC                & $591.29253$     & $617.14337$     & $581.03621$      & $513.34065$      & $512.46369$      \\
BIC                & $661.56110$     & $683.89852$     & $644.27793$      & $573.06894$      & $568.67855$      \\
Log Likelihood     & $-275.64627$    & $-289.57169$    & $-272.51811$     & $-239.67032$     & $-240.23184$     \\
Num. obs.          & $248$           & $248$           & $248$            & $248$            & $248$            \\
Censored           & $113$           & $113$           & $113$            & $113$            & $113$            \\
Observed           & $135$           & $135$           & $135$            & $135$            & $135$            \\
\hline
\multicolumn{6}{l}{\tiny{$^{***}p<0.001$; $^{**}p<0.01$; $^{*}p<0.05$}}
\end{tabular}
\end{small}
\caption{Selection Models for Fly Reel Prices}
\label{tab:tobit_5_sel}
\end{center}
\end{table}


\pagebreak

I revert back to Model 5 to analyze the differences
between country of manufacture, 
in the rightmost column of Table \ref{tab:tobit_5_sel}.
% 
First of all, this confirms that machined reels
are more valuable, with a coefficient of 0.76, instead of 0.63,
as was found in the separate linear regression model.
It also justifies the fact that only machined reels
are produced in the USA.
This could relate to some advantage in American production
technology or, rather, the outdated casting techniques
used for cheaper reels produced overseas.

In the model for American-made reels,
the coefficients on all three variables are statistically
the same as those in the separate linear regression model.
% 
For the fly reels made overseas, however,
\texttt{Width} replaces \texttt{Diameter} as a proxy for the size of the reels.
As a consequence, density is a relatively less valuable feature.
The value of sealed reels, however, is even greater,
after accounting for the selection of different production techniques
jointly with manufacturing location.
The coefficient on sealed reels jumps to 0.90, compared to 0.65
in the linear model.
This suggests a higher premium than indicated earlier,
once we also consider the choice of country of manufacture.
Similarly, the value of machined reels rises from 0.65 to 0.76
in the selection model, suggesting an even higher
for this design when produced overseas.


In conclusion, an American fly reel producer should not consider
producing cast reels, unless the purpose is to explore the change in value.
I suspect, however, that this outcome has been observed in
the past, so the older production techniques have been abandoned for a reason.
Machined reels are also more valuable when produced overseas,
so a company has to compare the difference in labor costs with the relative
cost of producing machined reels overseas.

Similarly, sealed reels produce a much higher premium overseas
than was originally estimated with the linear model.
Reels produced overseas should be sealed, unless the cost of
changing the manufacturing process would outweigh this premium.
Likewise for the American reels, except the premium is one third the size.

However it is measured, the size of a reel matters:
bigger reels are more valuable.
A manufacturer can compare the cost of materials with
the premiums attached to those dimensions when producing a reel.

Perhaps the largest difference is in the intercept term:
1.07 overseas vs 2.14 in the USA, double the size.
In the linear model, the intercepts were 2.41 vs 3.48,
which measures a similar percentage difference.
No matter how it is measured, there exists a substantial premium
for fly reels made in the USA.




%%%%%%%%%%%%%%%%%%%%%%%%%%%%%%%%%%%%%%%%
\end{document}
%%%%%%%%%%%%%%%%%%%%%%%%%%%%%%%%%%%%%%%%
